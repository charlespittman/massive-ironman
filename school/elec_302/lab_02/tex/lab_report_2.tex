%%%%%%%%%%%%%%%%%%%%%%%%%%%%%%%%%%%%%%%%%
% University/School Laboratory Report
% LaTeX Template
% Version 2.0 (4/12/12)
%
% This template has been downloaded from:
% http://www.latextemplates.com
%
% License:
% CC BY-NC-SA 3.0 (http://creativecommons.org/licenses/by-nc-sa/3.0/)
%
% Original header:
%
% This is a LaTeX version of the sample laboratory report
% from Virginia Tech's copyrighted 08-09 CHEM 1045/1046 lab manual.
% Reproduction of this one appendix section for academic purposes
% should fall under fair use.
%
%%%%%%%%%%%%%%%%%%%%%%%%%%%%%%%%%%%%%%%%%

\documentclass{article}
\usepackage{graphicx}
\usepackage{mathtools}

\title{ELEC 302-81\\ Lab 2\\ Transformer Fundamentals} % Title
% \author{John \textsc{Smith}} % Author name
\date{\today} % Specify a date for the report

\begin{document}

\maketitle

\begin{center}
  \begin{tabular}{lr}
    Date Performed: & January 28, 2013 \\
    Partners: & Rawley Dent \\
              & Charles Pittman \\
    Instructor: & Dr. Weatherford
  \end{tabular}
\end{center}

\pagebreak

%\setlength\parindent{0pt} % Removes all indentation from paragraphs

\section{Purpose of Experiment}
In this experiment, an RLC circuit was modeled on an EMS workstation.  The
capacitance was varied for two different inductance values. The circuit was
then analyzed to obtain measured values for the circuit power factor, real
power, and apparent power. These measured results were then compared to the
theoretical values calculated beforehand. Familiarization with the EMS
workstation was also obtained.

\section{Circuits Tested}
\begin{figure}[h]
  \begin{center}
    \includegraphics[width=.8\textwidth]{circuit_01}
    \caption{Single Phase Transformer Circuit for part }
    \label{fig:circuit_01}
  \end{center}
\end{figure}

\begin{figure}[h]
  \begin{center}
    \includegraphics[width=.8\textwidth]{circuit_02}
    \caption{}
    \caption{Single Phase Transformer Circuit for part two}
    \label{fig:circuit_02}
  \end{center}
\end{figure}

\begin{figure}[h]
  \begin{center}
    \includegraphics[width=.8\textwidth]{circuit_03}
    \caption{Single Phase Transformer Circuit for part three}
    \label{fig:circuit_03}
  \end{center}
\end{figure}

\begin{table}[h]
  \begin{center}
    \begin{tabular}{ccc}
      \hline
      R & L & C \\
      $\Omega$ & H & $\mu$F \\
      \hline
      1200 & 0.8 & --- \\ 1200 & 0.8 & 2.2 \\ 1200 & 0.8 & 4.4 \\
      1200 & 0.8 & 8.8 \\ 1200 & 1.6 & --- \\ 1200 & 1.6 & 2.2 \\
      1200 & 1.6 & 4.4 \\ 1200 & 1.6 & 8.8 \\
      \hline
    \end{tabular}
    \caption{RLC Values for circuit in Figure~\ref{circuit_01}}
    \label{tab:testc_dat}
  \end{center}
\end{table}

\section{Procedure}
At the EMS workstation, the main power switch of the Power Supply was verified
to be OFF, and the voltage control knob was verified to be completely
counterclockwise. The voltmeter selector switch was set to position 4-N. The
RLC circuit shown in Figure~\ref{fig:circuit_xx} was modeled with the capacitor
left out at first. The elements labeled E$_1$ and I$_1$ on
Figure~\ref{fig:circuit_xx} referred to the ammeter and voltmeter Data Acquisition
Interface (DAI) connections. The DAI 24-V supply was connected to the main
Power Supply, and the DAI USB cable was connected to the PC workstation. The
main power switch of the Power Supply was switched to ON.

On the PC, the Lab-Volt Data Acquisition Management application was started,
and the file pertaining to the experiment being worked was opened. Three
windows (metering, oscilloscope, and phasor analyzer) were verified to open up
along with the experiment file. The three windows were set to continuously
refresh.

The supply voltage was adjusted to read 24-V. Verification was obtained by
monitoring the EMS analog voltmeter and the digital metering window on the {PC}.
The load voltage E$_1$, load current I$_1$, the real power consumed by the
circuit, and the phase angle were recorded. In the Phasor Analyzer window,
E$_1$ was selected as the reference phasor. The voltage control knob was then
turn completely counterclockwise and the main power switch was set to OFF.

The EMS workstation was then reconfigured to include the capacitance.  The
subsequent values included in Table~\ref{tab:meas_data} were then measured.

\section{Calc}
\begin{table}[h]
  \begin{center}
    \begin{tabular}{cccccccccc}
      \hline
      R & L & C & I$_1$ & E$_1$ & P & $\theta$ & S & Q & p.f. \\
      $\Omega$ & H & $\mu$F & A & V & W & $\circ$ & VA & VAR & \\
      \hline
      1200 & 0.8 & --- & 0.210 & 60 & 4.56 &  68.77 & 12.58 & 11.73 & 0.36 \\
      1200 & 0.8 & 2.2 & 0.164 & 60 & 4.56 &  62.48 &  9.86 &  8.75 & 0.46 \\
      1200 & 0.8 & 4.4 & 0.122 & 60 & 4.56 &  51.65 &  7.34 &  5.76 & 0.62 \\
      1200 & 0.8 & 8.8 & 0.076 & 60 & 4.56 &  -2.67 &  4.56 & -0.21 & 1.00 \\
      1200 & 1.6 & --- & 0.114 & 60 & 3.39 &  60.27 &  6.84 &  5.94 & 0.50 \\
      1200 & 1.6 & 2.2 & 0.075 & 60 & 3.39 &  41.06 &  4.50 &  2.96 & 0.75 \\
      1200 & 1.6 & 4.4 & 0.057 & 60 & 3.39 &  -0.50 &  3.39 & -0.03 & 1.00 \\
      1200 & 1.6 & 8.8 & 0.115 & 60 & 3.39 & -60.51 &  6.89 & -6.00 & 0.49 \\
      \hline
    \end{tabular}
    \caption{Calculated Data}
    \label{tab:calc_data}
  \end{center}
\end{table}

\[\mathbf{P} = \mathbf{VI}\cos\theta \]
\[\mathbf{Q} = \mathbf{VI}\sin\theta\]
\[\mathbf{S} = \mathbf{VI}^*\]
\[\mathbf{V} = \mathbf{IZ}\]
\[\text{p.f.} = \cos\theta\]

\section{Results}
\begin{table}[h]
  \begin{center}
    \begin{tabular}{cccccccccc}
      \hline
      R & L & C & I$_1$ & E$_1$ & P & $\theta$ & S & Q & p.f. \\
      $\Omega$ & H & $\mu$F & A & V & W & $\circ$ & VA & VAR & \\
      \hline
      1200 & 0.8 & --- & 0.206 & 60.9 & 4.53 &  68.0 & 12.55 & 11.21 & 0.37 \\
      1200 & 0.8 & 2.2 & 0.158 & 60.9 & 4.56 &  60.9 &  9.62 &  8.19 & 0.49 \\
      1200 & 0.8 & 4.4 & 0.117 & 60.9 & 4.59 &  49.0 &  7.13 &  5.28 & 0.66 \\
      1200 & 0.8 & 8.8 & 0.081 & 61.0 & 4.65 &  -4.4 &  4.94 & -0.36 & 1.00 \\
      1200 & 1.6 & --- & 0.116 & 61.0 & 3.94 &  55.4 &  7.08 &  0.37 & 1.00 \\
      1200 & 1.6 & 2.2 & 0.079 & 61.0 & 3.96 &  32.8 &  4.82 &  2.55 & 0.84 \\
      1200 & 1.6 & 4.4 & 0.067 & 61.0 & 3.99 &  -6.6 &  4.09 & -0.46 & 1.00 \\
      1200 & 1.6 & 8.8 & 0.124 & 61.2 & 4.05 & -57.4 &  7.60 & -6.33 & 0.54 \\
      \hline
    \end{tabular}
    \caption{Experimental Data}
    \label{tab:meas_data}
  \end{center}
\end{table}

\section{Comparison with Theoretical: Percent Deviation}
\begin{table}[h]
  \begin{center}
    \begin{tabular}{cccccccccc}
      \hline
      R & L & C & I$_1$ & E$_1$ & P & $\theta$ & S & Q & p.f. \\
      \hline
      0.0 & 0.0 & --- & 1.9 & 1.5 & 0.7 & 1.1  & 0.2 & 4.4  & 2.8 \\
      0.0 & 0.0 & 0.0 & 3.7 & 1.5 & 0.0 & 2.5  & 2.4 & 6.4  & 6.5 \\
      0.0 & 0.0 & 0.0 & 4.1 & 1.5 & 0.7 & 5.1  & 2.9 & 8.3  & 6.5 \\
      0.0 & 0.0 & 0.0 & 6.6 & 1.7 & 2.0 & 65   & 8.3 & 71   & 0.0 \\
      0.0 & 0.0 & --- & 1.8 & 1.7 & 16  & 8.1  & 3.5 & 94   & 100 \\
      0.0 & 0.0 & 0.0 & 5.3 & 1.7 & 17  & 20   & 7.1 & 14   & 12  \\
      0.0 & 0.0 & 0.0 & 18  & 1.7 & 18  & 1220 & 21  & 1433 & 0.0 \\
      0.0 & 0.0 & 0.0 & 7.8 & 2.0 & 19  & 5.1  & 10  & 5.5  & 10  \\
    \end{tabular}
  \end{center}
\end{table}

\[\text{\%deviation} = \frac{measured - theoretical}{theoretical}\ \times \text{100\%}\]

\section{Conclusions}
The effects of different levels of capacitance were observed by conducting this
experiment. The theoretical values for real power, apparent power, and the
phase angle were calculated. Then, the experiment was conducted to verify the
effects of a parallel RLC circuit. These measured results were differed from
the theoretical results because of one underlying reason. The switch modeling
the inductance at the EMS workstation contained an internal impedance.  This
added impedance caused the calculated results to differ from the measured.
Also, one can see that the voltage measured at E$_1$ fluctuating throughout the
experiment. This fluctuation caused added variations between measured and
theoretical.

It was noted that row 8 of Tables~\ref{calc_data} and~\ref{meas_data} contained
a negative angle for the phase impedance. This signified a highly capacitative
load where the phase current lead the phase voltage. All other loads were
inductive, except in rows 4 and 7, where the load was slightly capacitative. In
these rows, 4 and 7, the power factor recorded as 1.0 signifying an inductive
load efficiently corrected by a capacitance.

\end{document}
