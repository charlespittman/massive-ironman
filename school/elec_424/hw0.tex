% Created 2015-05-13 Wed 16:04
\documentclass[11pt]{article}
\usepackage[utf8]{inputenc}
\usepackage[T1]{fontenc}
\usepackage{fixltx2e}
\usepackage{graphicx}
\usepackage{longtable}
\usepackage{float}
\usepackage{wrapfig}
\usepackage{rotating}
\usepackage[normalem]{ulem}
\usepackage{amsmath}
\usepackage{textcomp}
\usepackage{marvosym}
\usepackage{wasysym}
\usepackage{amssymb}
\usepackage{hyperref}
\tolerance=1000
\author{Charles Pittman}
\date{\today}
\title{hw0}
\hypersetup{
  pdfkeywords={},
  pdfsubject={},
  pdfcreator={Emacs 24.5.1 (Org mode 8.2.10)}}
\begin{document}

\maketitle
\tableofcontents

From the Executive Summary

\section{How many key contributors were in the “metrology” working group?  How about in the “test and test equipment” group?}
\label{sec-1}
There are 26 in the "Metrology" group. There are 77 in the "Test and Test
Equipment".

\section{Just by glance, which group has the most key contributors?}
\label{sec-2}
The "Emerging Research Materials" has the most key contributors.

\section{Roughly, how many persons contributed to the roadmap in total?  From the entire document?}
\label{sec-3}
There are about 1,100 key contributors.

\section{Click around the roadmap, be awed by its depth and staggered by its detail. Then find some obscure but interesting fact to share with the class in a couple of paragraphs (no more than one page) of writing. Do not simply cut and paste, but put what interests you in your own words.}
\label{sec-4}
\begin{itemize}
\item Chemistry is \uline{way} more important than I thought.  "Emerging Research Materials" having so many contributors should've been a hint.  A large chunk of the research and difficulties revolves around finding new materials with different properties.
\item The testing section is titled "Test and Test Equippment" \footnote{\url{http://www.itrs.net/ITRS\%201999-2014\%20Mtgs,\%20Presentations\%20&\%20Links/2013ITRS/2013Chapters/2013Test_Summary.pdf}}
\end{itemize}
% Emacs 24.5.1 (Org mode 8.2.10)
\end{document}