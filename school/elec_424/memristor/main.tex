\author{Charles Pittman}
\date{\today}
\title{Memristors}

\documentclass[12pt]{article}

\begin{document}
\maketitle

%\begin{abstract}
%This is the paper's abstract \ldots
%\end{abstract}

\section{Introduction}
Electronics textbooks list three passive components: resistors, capacitors, and
inductors.  Resistors relate voltage to current, capacitors relate voltage to
charge, and inductors relate charge to magnetic flux.  Noting the symmetry (no
component relating magnetic flux to current), in 1971 Leon Chua published a
mathematical proof \cite{chua1971} that such a device should be possible.

As a function of time, the device gives a flux-charge relationship similar to
the current-voltage relationship of a resistor (i.e. its resistance changes
according to the charge passed through it).  The device remembers it's history,
which is where the term ``memristor'' comes from.  Active devices simulating
the behavior had existed, but the first passive version came from HP 37 years
later. \cite{strukov2008missing}

%\section{Implementation}
%\subsection{Titanium Dioxide}
HP's device consisted of two layers of titanium dioxide ($\mathrm{TiO_2}$), a
semiconductor, sandwiched between platinum plates with some oxygen atoms
removed from one of the layers to make it conductive ($\mathrm{TiO_{2-x}}$).
Voltage applied to the platinum plates would expand or contract the
$\mathrm{TiO_{2-x}}$ layer, changing the device's conductivity
\cite{williams2008we} (a switch, essentially).

\section{Applications}

\subsection{Non-Volatile Memory}\label{memory}
The device HP created was born out of a research group tasked with figuring out
what to do when transistors could be shrunk no further \cite{williams2008we}.
As component sizes shrink aligning masks during the die creation becomes more
difficult and probability of surface defects increase, together reducing yield
\cite{snider2008molecular}.  The team realized that redundancy could be used to
combat the reduced yield.  Inspired by another HP project, the Teramac
\cite{heath1998} massively-parallel computer, the cross-board latch would be
used: multiple switches are placed between an input and output line such that
any could trigger a connection.  Forming an array of these latches created a
storage device, and since memristors preserve their state no power is required
to preserve the data.

\subsection{Logic}\label{logic}
The extension from storage to logic is straightforward: for any operation a
given input will produce the same output, so a table of results can be created
for a list of inputs.  Memristors can implement logic functions as well; the
basic NAND gate requires one less component compared to transistors.  Besides
boolean logic memristors can be used to build the IMPLY conditional (p
$\Rightarrow $ q) \cite{DBLP:journals/corr/abs-1110-2074} which is useful in implementing fuzzy logic.

\subsection{Learning Circuits}\label{ai}
A 2009 study\cite{pershin2009} showed a simple memristive circuit was able to
anticipate voltage spikes when a steady pulse train was applied.  The
experiment was designed to emulate an earlier study \cite{saigusa2008} on
behavioral intelligence in slime mold.  In it the mold was able to predict
periodic changes to its environment.  A separate study
\cite{nakagaki2000intelligence} of the slime mold showed the single-cell
organisms capable of solving a maze via the shortest path; a network of memristors was later shown capable of the same \cite{pershin2011}.

%\subsection{Base-n Memory}
%In 2014 physicists at Trinity College Dublin created a memristor able to switch
%to six discrete states. \cite{hellemans_2014}

\newpage{}

\bibliographystyle{abbrv}
\bibliography{main}

\end{document}
This is never printed
