\author{Charles Pittman}
\date{\today}
\title{Memristors}

\documentclass[12pt]{article}

\begin{document}
\maketitle

\begin{abstract}
This is the paper's abstract \ldots
\end{abstract}

\section{Introduction}
Electronics textbooks list three passive components: resistors, capacitors, and
inductors.  Resistors relate voltage to current, capacitors relate voltage to
charge, and inductors relate charge to magnetic flux.  Noting the symmetry (no
component relating magnetic flux to current), in 1971 Leon Chua published a
mathematical proof \cite{chua1971} that such a device should be possible.

As a function of time, the device gives a flux-charge relationship similar to
the current-voltage relationship of a resistor (i.e. its resistance changes
according to the charge passed through it).  The device remembers it's history,
which is where the term ``memristor'' comes from.  Active devices simulating
the behavior had existed, but the first passive version came from HP 37 years
later.\cite{strukov2008missing}

\paragraph{Outline}
The remainder of this article is organized as follows.
Section~\ref{previous work} gives account of previous work.
Our new and exciting results are described in Section~\ref{results}.
Finally, Section~\ref{conclusions} gives the conclusions.

\section{Applications}
\subsection{Memory}\label{memory}
\subsection{Logic}\label{logic}

\section{Previous work}\label{previous work}
A much longer \LaTeXe{} example was written by Gil~\cite{Gil:02}.

\section{Results}\label{results}
In this section we describe the results.

\section{Conclusions}\label{conclusions}
We worked hard, and achieved very little.

\bibliographystyle{abbrv}
\bibliography{main}

\end{document}
This is never printed
