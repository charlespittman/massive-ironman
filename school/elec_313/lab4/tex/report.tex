% Trying to break the document up a bit.  This command simply inserts the contents of the file at this point.  It contains the document license, preamble, and title page: things that aren't likely to change more than once.  This can be used to separate discrete parts of a document into files that're easier to edit at one time.
%%%%%%%%%%%%%%%%%%%%%%%%%%%%%%%%%%%%%%%%%%%%%%%%%%%%%%%%%%%%%%%%%%%%%%
% This layout was adapted from one found at latextemplates.com which
% was adapted from another.
%
% License: CC BY-NC-SA 3.0
% (http://creativecommons.org/licenses/by-nc-sa/3.0/)
%
% Original header:
%
% This is a LaTeX version of the sample laboratory report from
% Virginia Tech's copyrighted 08-09 CHEM 1045/1046 lab manual.
% Reproduction of this one appendix section for academic purposes
% should fall under fair use.
%
%%%%%%%%%%%%%%%%%%%%%%%%%%%%%%%%%%%%%%%%%%%%%%%%%%%%%%%%%%%%%%%%%%%%%%

\documentclass{article}

\usepackage{graphicx}
% \usepackage[acronym]{glossaries} % Lets us use acronyms
\usepackage{multicol}
\usepackage{amsmath}
\usepackage{siunitx} % SI units in math mode
%\usepackage[usenames]{color}
\usepackage{subcaption}

\author{}
\title{ELEC-313 \\ Lab 9: Common-Emitter Transistor Amplifier\\ }
\date{\today}

% \loadglsentries{acronyms} % Actually loads 'acronyms.tex'
% \makeglossaries

\begin{document}

\maketitle

\begin{center}
  \begin{tabular}{lr}
    Date Performed: & November 20, 2013 \\
    Partners:       & Charles Pittman    \\
    & Stephen Wilson     \\
  \end{tabular}
\end{center}

\newpage

%\tableofcontents
%\listoffigures
%\listoftables
%\newpage

% Number the enumerate environment (unordered lists) by letter:
% \renewcommand{\labelenumi}{\alph{enumi}.}

\section{Objective}
\label{sec:objective}

The objective is to construct, measure, and observe the operation of a DC motor driver.

\section{Equipment}
\label{sec:equipment}

\begin{tabular}{ll}
  \centering
  Compact L298 Motor Driver Kit & Function generator: HP 33120A \\
  \SI{6}{V} DC Motor            & Multimeter:        \\
  Power supply: HP E3631A       & Oscilloscope: Agilent 54622D  \\
\end{tabular}

\section{Schematics}
\label{sec:schematics}

% \begin{figure}[hbtp]
%   \centering
%   \begin{subfigure}[b]{0.6\textwidth}
%     \includegraphics[width=\textwidth]{volt_rect}
%     \caption{\label{fig:volt_rect} Voltage rectifier circuit.}
%   \end{subfigure}%
%   ~
%   \begin{subfigure}[b]{0.4\textwidth}
%     \includegraphics[width=\textwidth]{volt_reg}
%     \caption{\label{fig:volt_reg} Voltage regulator circuit.}
%   \end{subfigure}
%   \caption{\label{fig:circuits_tested} Circuits used in this lab.}
% \end{figure}

\section{Procedure}
\label{sec:procedure}

\subsection{Part One}
\label{sec:part_one}

% First, the capacitance of a \SI{1}{\micro\farad} capacitor was measured and  recorded in Table~\ref{tab:cap} along with the percent difference calculated (Eq~\ref{eq:percent_diff}).  Then the circuit shown in Figure~\ref{fig:volt_rect} was constructed on a breadboard.  The voltage source ($V_S$) and \SI{50}{\ohm} resistor seen in Figure~\ref{fig:volt_rect} represented the function generator used.  Using a resistive decade box, the load resistance ($R_L$) was initially set to \SI{10}{\kilo\ohm}.  Then, the first channel of the oscilloscope was set to measure the output of the function generator and the second channel was set to measure the voltage across the load ($V_L$).  The function generator was programmed to produce a sine wave of \SI{1}{V_{peak}} at \SI{400}{\hertz}, measured with the oscilloscope.  $V_{max}$, $V_{min}$, and DC voltage across $V_L$ was measured with a multimeter and recorded in Table~\ref{tab:rect_vp_vdc} along with the ripple voltage $V_r$ (Eq~\ref{eq:ripple}).  Finally, the source voltage was then left at \SI{5}{V_{peak}}, and $R_L$ was adjusted to \SI{1}{\kilo\ohm} and \SI{100}{\kilo\ohm} and each $V_{DC}$, $V_{max}$, $V_{min}$, and $V_r$ were measured and recorded in Table~\ref{tab:load_v_ripple}.

\subsection{Part Two}
\label{sec:part_two}

% First, a \SI{47}{\ohm} resistor was measured and its percent difference from the nominal resistance value was calculated and both values were recorded in Table~\ref{tab:res}.  The circuit shown in Figure~\ref{fig:volt_reg} was constructed on a breadboard, the HP power supply for the voltage source ($V_S$) set to 9 V, and a decade resistance box as the load resistance ($R_L$) set to \SI{1}{\kilo\ohm}.  Then the load voltage ($V_L$) and load current ($I_L$) was measured and recorded in Table~\ref{tab:volt_reg_meas}, for the $R_L$ of \SI{1}{\kilo\ohm}, \SI{330}{\ohm}, and \SI{100}{\ohm}.  The Zener diode was removed and the open-circuit voltage ($V_{OC}$) was measured and recorded in Table~\ref{tab:volt_reg_meas}, again for the $R_L$ of \SI{1}{\kilo\ohm}, \SI{330}{\ohm}, and \SI{100}{\ohm}.  The Zener diode was added back into the circuit and, using the $R_L$ of \SI{1}{\kilo\ohm}, \SI{330}{\ohm}, and \SI{100}{\ohm}, the point at which the source drops out of regulation ($V_S$ drop) was determined and recorded in Table~\ref{tab:volt_reg_meas} by sweeping the $V_S$.  Finally, the $R_L$ was removed ($R_L$=$\infty$), $V_S$ was set to 9 V, and the voltage across the Zener diode (also $V_L$) was measured and recorded in Table~\ref{tab:volt_reg_meas}.

\section{Results}
\label{sec:results}

\begin{table}[hbtp]
  \centering
  \begin{tabular}{ccc|ccc}
    Enable & \si{L_1} & \si{L_2} & \si{V_{out}} & LED & Motor \\
    \hline
    L & L & L & \SI{-0.01}{V} & off & off \\
    L & L & H & \SI{-0.01}{V} & off & off \\
    L & H & L & \SI{-0.01}{V} & off & off \\
    L & H & H & \SI{-0.01}{V} & off & off \\
    H & L & L & \SI{-0.18}{V} & off & off \\
    H & L & H & \SI{+5.7}{V} & red & CW \\
    H & H & L & \SI{+5.5}{V} & green & CCW \\
    H & H & H & \SI{+0.01}{V} & both & off \\
  \end{tabular}
  \caption{\label{tab:logic} Logic Table}
\end{table}

\begin{table}[hbtp]
  \centering
  \begin{tabular}{cc}
    Duty Cycle & Vout \\
    \hline
    20\% & \SI{-3.01}{V} \\
    30\% & \SI{-3.39}{V} \\
    40\% & \SI{-3.76}{V} \\
    50\% & \SI{-4.13}{V} \\
    60\% & \SI{-4.49}{V} \\
    70\% & \SI{-4.84}{V} \\
    80\% & \SI{-5.19}{V} \\
  \end{tabular}
  \caption{\label{tab:duty} Pulse-width modulation}
\end{table}

\section{Comparison of Results}
\label{sec:comp_of_res}

% The PSpice computed values of $V_{OC}$ shown in Table~\ref{tab:volt_reg_calc}, were very close to the measured $V_{OC}$ values (Table~\ref{tab:volt_reg_calc}), and were at most only 3.327\% different (Table~\ref{tab:volt_reg_diff}).  Also, the PSpice computed $V_S$ drop (Table~\ref{tab:volt_reg_calc}) was very close to the measured $V_S$ drop (Table~\ref{tab:volt_reg_meas}) and the percent difference was at most only 1.7\% (Table~\ref{tab:volt_reg_diff}).

\section{Conclusion}
\label{sec:conclusion}
%%% There's no reason to refer to equations after the first time in such a short document.  Once in the procedure should be good enough.  :D
% As seen in Figure~\ref{fig:rect_vp_vdc}, as the peak voltage of the rectifier circuit increases, the rectifier DC voltage increases.  Also, as load resistance is increased on the rectifier, it decreases the percent ripple as shown in Table~\ref{tab:load_v_ripple}.  This was probably because the increase of resistance reduced the amount of current that could be dissipated from the capacitor over the same amount of time before it was “charged up” again.  Additionally, the percent ripple would likely decrease as input frequency increases because there would be a smaller time interval for the capacitor to discharge.

% \section{Equations}
% \label{sec:equations}

% % LaTeX sees blank lines as a start of another paragraph.  To avoid
% % unnecessary vertical spaces between equations, and still visually
% % separate in source, put a comment between them.
% %
% \begin{equation}
%   \label{eq:percent_diff}
%   \%_{diff} = \frac{|nominal - measured|}{nominal}\times 100\%
% \end{equation}
% %
% \begin{equation}
%   \label{eq:ripple}
%   V_r = V_{max} - V_{min}
% \end{equation}
% %
% \begin{equation}
%   \label{eq:volt_reg}
%   \%_{reg} = \frac{V_{load} - V_{no load}}{V_{no load}}\times 100\%
% \end{equation}

\end{document}
