%%%%%%%%%%%%%%%%%%%%%%%%%%%%%%%%%%%%%%%%%%%%%%%%%%%%%%%%%%%%%%%%%%%%%%
% This layout was adapted from one found at latextemplates.com which
% was adapted from another.
%
% License: CC BY-NC-SA 3.0
% (http://creativecommons.org/licenses/by-nc-sa/3.0/)
%
% Original header:
%
% This is a LaTeX version of the sample laboratory report from
% Virginia Tech's copyrighted 08-09 CHEM 1045/1046 lab manual.
% Reproduction of this one appendix section for academic purposes
% should fall under fair use.
%
%%%%%%%%%%%%%%%%%%%%%%%%%%%%%%%%%%%%%%%%%%%%%%%%%%%%%%%%%%%%%%%%%%%%%%

\documentclass{article}

\usepackage{graphicx}
%\usepackage[acronym]{glossaries} % Lets us use acronyms
\usepackage{multicol}
\usepackage{amsmath}
\usepackage{siunitx} % SI units in math mode
\usepackage{subcaption}

\author{}
\title{ELEC-313 \\ Lab 3: Diode Circuits\\ }
\date{\today}

%\loadglsentries{acronyms} % Actually loads 'acronyms.tex'
%\makeglossaries

\begin{document}

\maketitle

\begin{center}
  \begin{tabular}{lr}
    Date Performed: & September 25, 2013 \\
    Partners:       & Charles Pittman    \\
                    & Stephen Wilson     \\
  \end{tabular}
\end{center}

\newpage

\tableofcontents
\listoffigures
\listoftables

\newpage

% Number the enumerate environment (unordered lists) by letter:
\renewcommand{\labelenumi}{\alph{enumi}.}

\section{Objective}
\label{sec:objective}

The objective is to construct, measure, and observe the behavior of two common diode circuits: a voltage rectifier, and a voltage regulator.

\section{Equipment}
\label{sec:equipment}

\begin{tabular}{ll}
  \centering
  Diode: 1N4007                          & Power supply: HP E3631A       \\
  Zener diode: 1N5231                    & Function generator: HP 33120A \\
  Resistors: \SI{47}{\ohm}               & Multimeter: Fluke 8010A       \\
  Capacitor: \SI{1}{\micro\farad}        & Oscilloscope: Agilent 54622D  \\
  Resistive decade box: HeathKit IN-3117 &                               \\
\end{tabular}

\section{Schematics}
\label{sec:schematics}

\begin{figure}[hbtp]
  \centering
  \begin{subfigure}[b]{0.6\textwidth}
    \includegraphics[width=\textwidth]{volt_rect}
    \caption{\label{fig:volt_rect} Voltage rectifier circuit.}
  \end{subfigure}%
  ~
  \begin{subfigure}[b]{0.4\textwidth}
    \includegraphics[width=\textwidth]{volt_reg}
    \caption{\label{fig:volt_reg} Voltage regulator circuit.}
  \end{subfigure}
  \caption{\label{fig:circuits_tested} Circuits used in this lab.}
\end{figure}

\section{Procedure}
\label{sec:procedure}

\subsection{Voltage Rectifier}
\label{sec:rectifier}

First, the capacitance of a \SI{1}{\micro\farad} capacitor was measured and  recorded in Table~\ref{tab:cap} along with the percent difference calculated (Eq~\ref{eq:percent_diff}).  Then the circuit shown in Figure~\ref{fig:volt_rect} was constructed on a breadboard.  The voltage source ($V_S$) and \SI{50}{\ohm} resistor seen in Figure~\ref{fig:volt_rect} represented the function generator used.  Using a resistive decade box, the load resistance ($R_L$) was initially set to \SI{10}{\kilo\ohm}.  Then, the first channel of the oscilloscope was set to measure the output of the function generator and the second channel was set to measure the voltage across the load ($V_L$).  The function generator was programmed to produce a sine wave of \SI{1}{V_{peak}} at \SI{400}{\hertz}, measured with the oscilloscope.  $V_{max}$, $V_{min}$, and DC voltage across $V_L$ was measured with a multimeter and recorded in Table~\ref{tab:rect_vp_vdc} along with the ripple voltage $V_r$ (Eq~\ref{eq:ripple}).  Finally, the source voltage was then left at \SI{5}{V_{peak}}, and $R_L$ was adjusted to \SI{1}{\kilo\ohm} and \SI{100}{\kilo\ohm} and each $V_{DC}$, $V_{max}$, $V_{min}$, and $V_r$ were measured and recorded in Table~\ref{tab:load_v_ripple}.

\subsection{Voltage Regulator}
\label{sec:volt_reg}

First, a \SI{47}{\ohm} resistor was measured and its percent difference from the nominal resistance value was calculated and both values were recorded in Table~\ref{tab:res}.  The circuit shown in Figure~\ref{fig:volt_reg} was constructed on a breadboard, the HP power supply for the voltage source ($V_S$) set to 9 V, and a decade resistance box as the load resistance ($R_L$) set to \SI{1}{\kilo\ohm}.  Then the load voltage ($V_L$) and load current ($I_L$) was measured and recorded in Table~\ref{tab:volt_reg_meas}, for the $R_L$ of \SI{1}{\kilo\ohm}, \SI{330}{\ohm}, and \SI{100}{\ohm}.  The Zener diode was removed and the open-circuit voltage ($V_{OC}$) was measured and recorded in Table~\ref{tab:volt_reg_meas}, again for the $R_L$ of \SI{1}{\kilo\ohm}, \SI{330}{\ohm}, and \SI{100}{\ohm}.  The Zener diode was added back into the circuit and, using the $R_L$ of \SI{1}{\kilo\ohm}, \SI{330}{\ohm}, and \SI{100}{\ohm}, the point at which the source drops out of regulation ($V_S$ drop) was determined and recorded in Table~\ref{tab:volt_reg_meas} by sweeping the $V_S$.  Finally, the $R_L$ was removed ($R_L$=$\infty$), $V_S$ was set to 9 V, and the voltage across the Zener diode (also $V_L$) was measured and recorded in Table~\ref{tab:volt_reg_meas}.

\section{Results}
\label{sec:results}

\begin{table}[hbtp]
  \centering
  \begin{tabular}{ccc}
    Nominal             & Measured            & Difference \\
    (\si{\micro\farad}) & (\si{\micro\farad}) &            \\
    \hline
    1                   & 0.938               & 6.2\%      \\
  \end{tabular}
  \caption{\label{tab:cap} Percent difference of capacitor in rectifier circuit.}
\end{table}

\begin{table}[hbtp]
  \centering
  \begin{tabular}{cccccc}
    $V_S$           & $V_{max}$ & $V_{min}$ & $V_r$    & $V_{DC}$ & Ripple \\
    (\si{V_{peak}}) & (\si{V})  & (\si{V})  & (\si{V}) & (\si{V}) &        \\
    \hline
    1               & 0.488     & 0.369     & 0.119    & 0.429    & 24.4\% \\
    2               & 1.41      & 1.10      & 0.310    & 1.26     & 22.0\% \\
    3               & 2.39      & 1.88      & 0.510    & 2.14     & 21.3\% \\
    4               & 3.31      & 2.38      & 0.930    & 2.85     & 28.1\% \\
    5               & 4.25      & 3.19      & 1.06     & 3.72     & 24.9\% \\
  \end{tabular}
  \caption{\label{tab:rect_vp_vdc} AC input vs. DC output of rectifier circuit, where $R_L=\SI{10}{\kilo\ohm}$.}
\end{table}

\begin{table}[hbtp]
  \centering
  \begin{tabular}{cccccc}
    $R_L$       & $V_{max}$ & $V_{min}$ & $V_r$    & $V_{DC}$ & Ripple  \\
    (\si{\ohm}) & (\si{V})  & (\si{V})  & (\si{V}) & (\si{V}) &         \\
    \hline
    1k          & 4.13      & 0.440     & 3.69     & 2.29     & 89.3\%  \\
    10k         & 4.25      & 3.19      & 1.06     & 3.72     & 24.9\%  \\
    100k        & 4.321     & 4.193     & 0.128    & 4.257    & 2.962\% \\
  \end{tabular}
  \caption{\label{tab:load_v_ripple} Effect of $R_L$ on DC output in rectifier circuit.}
\end{table}

\begin{figure}[hbtp]
  \centering
  % GNUPLOT: LaTeX picture
\setlength{\unitlength}{0.240900pt}
\ifx\plotpoint\undefined\newsavebox{\plotpoint}\fi
\sbox{\plotpoint}{\rule[-0.200pt]{0.400pt}{0.400pt}}%
\begin{picture}(1500,900)(0,0)
\sbox{\plotpoint}{\rule[-0.200pt]{0.400pt}{0.400pt}}%
\put(171.0,131.0){\rule[-0.200pt]{4.818pt}{0.400pt}}
\put(151,131){\makebox(0,0)[r]{ 0}}
\put(1419.0,131.0){\rule[-0.200pt]{4.818pt}{0.400pt}}
\put(171.0,212.0){\rule[-0.200pt]{4.818pt}{0.400pt}}
\put(151,212){\makebox(0,0)[r]{ 0.5}}
\put(1419.0,212.0){\rule[-0.200pt]{4.818pt}{0.400pt}}
\put(171.0,292.0){\rule[-0.200pt]{4.818pt}{0.400pt}}
\put(151,292){\makebox(0,0)[r]{ 1}}
\put(1419.0,292.0){\rule[-0.200pt]{4.818pt}{0.400pt}}
\put(171.0,373.0){\rule[-0.200pt]{4.818pt}{0.400pt}}
\put(151,373){\makebox(0,0)[r]{ 1.5}}
\put(1419.0,373.0){\rule[-0.200pt]{4.818pt}{0.400pt}}
\put(171.0,454.0){\rule[-0.200pt]{4.818pt}{0.400pt}}
\put(151,454){\makebox(0,0)[r]{ 2}}
\put(1419.0,454.0){\rule[-0.200pt]{4.818pt}{0.400pt}}
\put(171.0,534.0){\rule[-0.200pt]{4.818pt}{0.400pt}}
\put(151,534){\makebox(0,0)[r]{ 2.5}}
\put(1419.0,534.0){\rule[-0.200pt]{4.818pt}{0.400pt}}
\put(171.0,615.0){\rule[-0.200pt]{4.818pt}{0.400pt}}
\put(151,615){\makebox(0,0)[r]{ 3}}
\put(1419.0,615.0){\rule[-0.200pt]{4.818pt}{0.400pt}}
\put(171.0,695.0){\rule[-0.200pt]{4.818pt}{0.400pt}}
\put(151,695){\makebox(0,0)[r]{ 3.5}}
\put(1419.0,695.0){\rule[-0.200pt]{4.818pt}{0.400pt}}
\put(171.0,776.0){\rule[-0.200pt]{4.818pt}{0.400pt}}
\put(151,776){\makebox(0,0)[r]{ 4}}
\put(1419.0,776.0){\rule[-0.200pt]{4.818pt}{0.400pt}}
\put(171.0,131.0){\rule[-0.200pt]{0.400pt}{4.818pt}}
\put(171,90){\makebox(0,0){ 1}}
\put(171.0,756.0){\rule[-0.200pt]{0.400pt}{4.818pt}}
\put(330.0,131.0){\rule[-0.200pt]{0.400pt}{4.818pt}}
\put(330,90){\makebox(0,0){ 1.5}}
\put(330.0,756.0){\rule[-0.200pt]{0.400pt}{4.818pt}}
\put(488.0,131.0){\rule[-0.200pt]{0.400pt}{4.818pt}}
\put(488,90){\makebox(0,0){ 2}}
\put(488.0,756.0){\rule[-0.200pt]{0.400pt}{4.818pt}}
\put(647.0,131.0){\rule[-0.200pt]{0.400pt}{4.818pt}}
\put(647,90){\makebox(0,0){ 2.5}}
\put(647.0,756.0){\rule[-0.200pt]{0.400pt}{4.818pt}}
\put(805.0,131.0){\rule[-0.200pt]{0.400pt}{4.818pt}}
\put(805,90){\makebox(0,0){ 3}}
\put(805.0,756.0){\rule[-0.200pt]{0.400pt}{4.818pt}}
\put(964.0,131.0){\rule[-0.200pt]{0.400pt}{4.818pt}}
\put(964,90){\makebox(0,0){ 3.5}}
\put(964.0,756.0){\rule[-0.200pt]{0.400pt}{4.818pt}}
\put(1122.0,131.0){\rule[-0.200pt]{0.400pt}{4.818pt}}
\put(1122,90){\makebox(0,0){ 4}}
\put(1122.0,756.0){\rule[-0.200pt]{0.400pt}{4.818pt}}
\put(1281.0,131.0){\rule[-0.200pt]{0.400pt}{4.818pt}}
\put(1281,90){\makebox(0,0){ 4.5}}
\put(1281.0,756.0){\rule[-0.200pt]{0.400pt}{4.818pt}}
\put(1439.0,131.0){\rule[-0.200pt]{0.400pt}{4.818pt}}
\put(1439,90){\makebox(0,0){ 5}}
\put(1439.0,756.0){\rule[-0.200pt]{0.400pt}{4.818pt}}
\put(171.0,131.0){\rule[-0.200pt]{0.400pt}{155.380pt}}
\put(171.0,131.0){\rule[-0.200pt]{305.461pt}{0.400pt}}
\put(1439.0,131.0){\rule[-0.200pt]{0.400pt}{155.380pt}}
\put(171.0,776.0){\rule[-0.200pt]{305.461pt}{0.400pt}}
\put(30,453){\makebox(0,0){$V_{DC} (V)}}
\put(805,29){\makebox(0,0){$V_S (V_{peak})$}}
\put(805,838){\makebox(0,0){Peak Voltage vs. DC Voltage in Rectifier Circuit}}
\put(171,200){\usebox{\plotpoint}}
\multiput(171.00,200.58)(1.184,0.499){265}{\rule{1.046pt}{0.120pt}}
\multiput(171.00,199.17)(314.828,134.000){2}{\rule{0.523pt}{0.400pt}}
\multiput(488.00,334.58)(1.117,0.499){281}{\rule{0.993pt}{0.120pt}}
\multiput(488.00,333.17)(314.939,142.000){2}{\rule{0.496pt}{0.400pt}}
\multiput(805.00,476.58)(1.381,0.499){227}{\rule{1.203pt}{0.120pt}}
\multiput(805.00,475.17)(314.504,115.000){2}{\rule{0.601pt}{0.400pt}}
\multiput(1122.00,591.58)(1.133,0.499){277}{\rule{1.006pt}{0.120pt}}
\multiput(1122.00,590.17)(314.913,140.000){2}{\rule{0.503pt}{0.400pt}}
\put(171,200){\makebox(0,0){$+$}}
\put(488,334){\makebox(0,0){$+$}}
\put(805,476){\makebox(0,0){$+$}}
\put(1122,591){\makebox(0,0){$+$}}
\put(1439,731){\makebox(0,0){$+$}}
\put(171.0,131.0){\rule[-0.200pt]{0.400pt}{155.380pt}}
\put(171.0,131.0){\rule[-0.200pt]{305.461pt}{0.400pt}}
\put(1439.0,131.0){\rule[-0.200pt]{0.400pt}{155.380pt}}
\put(171.0,776.0){\rule[-0.200pt]{305.461pt}{0.400pt}}
\end{picture}

  \caption{\label{fig:rect_vp_vdc} AC input vs. DC output of rectifier circuit, where $R_L=\SI{10}{\kilo\ohm}$}
\end{figure}

\begin{table}[hbtp]
  \centering
  \begin{tabular}{ccc}
    Nominal     & Measured    & Difference \\
    (\si{\ohm}) & (\si{\ohm}) &            \\
    \hline
    47          & 46.52       & 1.13\%     \\
  \end{tabular}
  \caption{\label{tab:res} Percent difference of resistor in voltage regulator circuit.}
\end{table}

\begin{table}
  \centering
  \begin{tabular}{ccc}
    $R_L$       & $V_{OC}$ & $V_S$ Drop \\
    (\si{\ohm}) & (\si{V}) & (\si{V})   \\
    \hline
    100         & 6.12     & 7.5        \\
    330         & 7.88     & 5.8        \\
    1k          & 8.90     & 5.3        \\
  \end{tabular}
  \caption{\label{tab:volt_reg_calc} Calculated values for voltage regulator circuit}
\end{table}

\begin{table}
  \centering
  \begin{tabular}{cccccc}
    $R_L$       & $V_L$    & $I_L$     & $V_{OC}$ & $V_S$ Drop & Voltage    \\
    (\si{\ohm}) & (\si{V}) & (\si{mA}) & (\si{V}) & (\si{V})   & Regulation \\
    \hline
    100         & 5.163    & 50.9      & 6.10     & 7.5        & 5.02\%     \\
    330         & 5.318    & 15.62     & 7.87     & 5.9        & 4.03\%     \\
    1k          & 5.11     & 5.27      & 8.60     & 5.3        & 1.15\%     \\
    $\infty$    & 5.38     & ---       & ---      & ---        & ---        \\
  \end{tabular}
  \caption{\label{tab:volt_reg_meas} Measured values for voltage regulator circuit}
\end{table}

\begin{table}
  \centering
  \begin{tabular}{ccccc}
    $R_L$       & $V_{OC}$  & $V_S$ Drop \\
    (\si{\ohm}) & (\% diff) & (\% diff)  \\
    \hline
    100         & 0.359\%   & 0.0\%      \\
    330         & 0.102\%   & 1.7\%      \\
    1k          & 3.327\%   & 0.0\%      \\
  \end{tabular}
  \caption{\label{tab:volt_reg_diff} Comparison of values for voltage regulator circuit}
\end{table}
\section{Comparison of Results}
\label{sec:comp_of_res}

%%% The comments about what the experimenter may've done belong more in the conclusion ***Disagree - we're comparing results in this section***, if anywhere ***Agree, it was really a joke. we were so  close becuase I knew waht the values were supposed to be; not because I was able to tell from by sweeping te voltage back n' forht as explained in the procedures.***  Also, not really sure what you're trying to get across***It was a lame attempt at humor - see previous comment***. [I did lol, particularly at "peaked".  Puns are the best.]  The more likely reason(s) values were so close: testing equipment wasn't precise enough ***Disagree***, and/or/mostly we didn't do an actual plot around the knee of the curve ***Agree*** /it's a log function so values change rapidly around that point anyways.  The only thing we could've done better was to have more independent data-takers ***Unnecessary***.
The PSpice computed values of $V_{OC}$ shown in Table~\ref{tab:volt_reg_calc}, were very close to the measured $V_{OC}$ values (Table~\ref{tab:volt_reg_calc}), and were at most only 3.327\% different (Table~\ref{tab:volt_reg_diff}).  Also, the PSpice computed $V_S$ drop (Table~\ref{tab:volt_reg_calc}) was very close to the measured $V_S$ drop (Table~\ref{tab:volt_reg_meas}) and the percent difference was at most only 1.7\% (Table~\ref{tab:volt_reg_diff}).% (suggesting that the experimenter may have “peeked” at the computed results and subconsciously (or consciously) noticed the recorded value to be almost exactly the same in all three circumstances).

\section{Conclusion}
\label{sec:conclusion}
%%% There's no reason to refer to equations after the first time in such a short document.  Once in the procedure should be good enough.  :D
As seen in Figure~\ref{fig:rect_vp_vdc}, as the peak voltage of the rectifier circuit increases, the rectifier DC voltage increases.  Also, as load resistance is increased on the rectifier, it decreases the percent ripple as shown in Table~\ref{tab:load_v_ripple}.  This was probably because the increase of resistance reduced the amount of current that could be dissipated from the capacitor over the same amount of time before it was “charged up” again.  Additionally, the percent ripple would likely decrease as input frequency increases because there would be a smaller time interval for the capacitor to discharge.

As shown in Table~\ref{tab:volt_reg_meas}, the voltage regulation across the \SI{100}{\ohm} resistor is 5.02\% different than the \SI{5.38}{V} measured when the load resistor was removed from the circuit.  This shows that the Zener diode used in the experiment is not an ideal Zener.  In the regulator circuit, when $V_S$ is below the Zener diode voltage ($V_Z$), $V_{OC}$ is linearly related to $V_S$.  When $V_S$ is above $V_Z$, $V_{OC}$ is almost at a constant value approximately equal to $V_Z$.  When $V_S$ is “close” to $V_Z$, the relationship of $V_{OC}$ to $V_S$ is not linearly related and the simplified calculations we learned in class are less effective at computing the $V_{OC}$.

% Information to include:
%
%[X] Measured results of rectifier circuit in tabular form, and the graph from the data analysis section.

%[X] Measured results of the voltage regulator in tabular form and the % regulation.

%[X] Comparison of the computed open-circuit voltages (prelab–part B) and the measured ones (step 4) over the range of RL.

%[X] Comparison of the measured drop out voltage (step 5) with the computed values from the prelab–part C over the range of RL.

%[ ] Explain the relationship between frequency and ripple size in the rectifier.  Explain the relationships among dropout voltage, open-circuit voltage, and the Zener diode voltage.

\section{Equations}
\label{sec:equations}

% LaTeX sees blank lines as a start of another paragraph.  To avoid
% unnecessary vertical spaces between equations, and still visually
% separate in source, put a comment between them.
%
\begin{equation}
  \label{eq:percent_diff}
  \%_{diff} = \frac{|nominal - measured|}{nominal}\times 100\%
\end{equation}
%
\begin{equation}
  \label{eq:ripple}
  V_r = V_{max} - V_{min}
\end{equation}
%
\begin{equation}
  \label{eq:volt_reg}
  \%_{reg} = \frac{V_{load} - V_{no load}}{V_{no load}}\times 100\%
\end{equation}

\end{document}
