%%%%%%%%%%%%%%%%%%%%%%%%%%%%%%%%%%%%%%%%%%%%%%%%%%%%%%%%%%%%%%%%%%%%%%
% This layout was adapted from one found at latextemplates.com which
% was adapted from another.
%
% License: CC BY-NC-SA 3.0
% (http://creativecommons.org/licenses/by-nc-sa/3.0/)
%
% Original header:
%
% This is a LaTeX version of the sample laboratory report from
% Virginia Tech's copyrighted 08-09 CHEM 1045/1046 lab manual.
% Reproduction of this one appendix section for academic purposes
% should fall under fair use.
%
%%%%%%%%%%%%%%%%%%%%%%%%%%%%%%%%%%%%%%%%%%%%%%%%%%%%%%%%%%%%%%%%%%%%%%

\documentclass{article}

\usepackage{graphicx}
%\usepackage[acronym]{glossaries} % Lets us use acronyms
\usepackage{multicol}
\usepackage{amsmath}
\usepackage{siunitx} % SI units in math mode
\usepackage{subcaption}

\author{}
\title{ELEC-313 \\ Lab 3: Diode Circuits\\ }
\date{\today}

%\loadglsentries{acronyms} % Actually loads 'acronyms.tex'
%\makeglossaries

\begin{document}

\maketitle

\begin{center}
  \begin{tabular}{lr}
    Date Performed: & September 25, 2013 \\
    Partners: & Charles Pittman \\
    & Stephen Wilson \\
  \end{tabular}
\end{center}

\newpage

\tableofcontents
\listoffigures
\listoftables

\newpage

% Number the enumerate environment (unordered lists) by letter:
\renewcommand{\labelenumi}{\alph{enumi}.}

\section{Objective}
\label{sec:objective}

The objective is to construct, measure, and observe the behavior of two common diode circuits: a voltage rectifier, and a voltage regulator.

\section{Equipment}
\label{sec:equipment}

\begin{tabular}{ll}
  \centering
  Diode: 1N4007 & Power supply: HP E3631A \\
  Zener diode: 1N5231 & Function generator: HP 33120A \\
  Resistors: \SI{47}{\ohm} & Multimeter: Fluke 8010A \\
  Capacitor: \SI{1}{\micro\farad} & Oscilloscope: Agilent 54622D \\
  Resistive decade box: HeathKit IN-3117 \\
\end{tabular}

\section{Schematics}
\label{sec:schematics}

\begin{figure}[hbtp]
  \centering
  \begin{subfigure}[b]{0.6\textwidth}
    \includegraphics[width=\textwidth]{volt_rect}
    \caption{\label{fig:volt_rect} Voltage rectifier circuit.}
  \end{subfigure}%
  ~
  \begin{subfigure}[b]{0.4\textwidth}
    \includegraphics[width=\textwidth]{volt_reg}
    \caption{\label{fig:volt_reg} Voltage regulator circuit.}
  \end{subfigure}
  \caption{\label{fig:circuits_tested} Circuits used in this lab.}
\end{figure}

\section{Procedure}
\label{sec:procedure}

\subsection{Rectifier}
\label{sec:rectifier}

% The circuit in Figure~\ref{fig:circuit1} was constructed with $R =
% \SI{470}{\ohm}$ and the power supply as $V_s$.  The actual resistance
% was measured with one a multimeter and recorded in
% Table~\ref{tab:percent_diff} along with the percent difference calculated
% (Eq~\ref{eq:percent_diff}).  Next, the multimeters were used to
% measure voltage across and current through the diode ($V_d$ and $I_d$,
% respectively) while $V_s$ was swept from \SI{-5}{V} to \SI{+10}{V}.
% The step size from \SI{-5}{V} to \SI{0}{V} and from \SI{5}{V} to
% \SI{10}{V} was \SI{0.5}{V}, and \SI{0.25}{V} from \SI{0}{V} to
% \SI{5}{V}.  These values were recorded in Table~\ref{tab:part_a} and
% plotted in Figure~\ref{fig:combined_graph}.

\subsection{Voltage Regulator}
\label{sec:volt_reg}

% The circuit in Figure~\ref{fig:circuit1} was constructed with the
% resistive decade box as $R$ and the power supply as $V_s$.  The
% multimeters were again used to measure diode voltage ($V_d$) and
% current ($I_d$).  This time $V_s$ was held at \SI{10}{V} and $R$
% varied: \SI{200}{\ohm}, \SI{500}{\ohm}, \SI{1}{\kilo\ohm},
% \SI{2}{\kilo\ohm}, \SI{5}{\kilo\ohm}, \SI{10}{\kilo\ohm},
% \SI{20}{\kilo\ohm}, \SI{50}{\kilo\ohm}, \SI{100}{\kilo\ohm}.  These
% values were recorded in Table~\ref{tab:part_b} and plotted in
% Figure~\ref{fig:combined_graph}.

\section{Results}
\label{sec:results}

\begin{table}[hbtp]
  \centering
  \begin{tabular}{ccc}
    Nominal & Measured & Difference \\
    (\si{\micro\farad}) & (\si{\micro\farad}) & \\
    \hline
    1 & 0.938 & 6.2\% \\
  \end{tabular}
  \caption{\label{tab:cap} Percent difference of capacitor in rectifier circuit.}
\end{table}

\begin{table}[hbtp]
  \centering
  \begin{tabular}{cccccc}
    $V_S$ & $V_{max}$ & $V_{min}$ & $V_r$ & $V_{DC}$ & Ripple \\
    (\si{V_{peak}}) & (\si{V}) & (\si{V}) & (\si{V}) & (\si{V}) & \\
    \hline
    1 & 0.488 & 0.369 & 0.119 & 0.429 & 24.4\% \\
    2 & 1.41 & 1.10 & 0.310 & 1.26 & 22.0\% \\
    3 & 2.39 & 1.88 & 0.510 & 2.14 & 21.3\% \\
    4 & 3.31 & 2.38 & 0.930 & 2.85 & 28.1\% \\
    5 & 4.25 & 3.19 & 1.06 & 3.72 & 24.9\% \\
  \end{tabular}
  \caption{\label{tab:rect_vp_vdc} AC input vs. DC output of rectifier circuit, where $R_L=\SI{10}{\kilo\ohm}$.}
\end{table}

\begin{table}[hbtp]
  \centering
  \begin{tabular}{cccccc}
    $R_L$ & $V_{max}$ & $V_{min}$ & $V_r$ & $V_{DC}$ & Ripple \\
    (\si{\ohm}) & (\si{V}) & (\si{V}) & (\si{V}) & (\si{V}) & \\
    \hline
    1k & 4.13 & 0.440 & 3.69 & 2.29 & 89.3\% \\
    10k & 4.25 & 3.19 & 1.06 & 3.72 & 24.9\% \\
    100k & 4.321 & 4.193 & 0.128 & 4.257 & 2.962\% \\
  \end{tabular}
  \caption{\label{tab:load_v_ripple} Effect of $R_L$ on DC output in rectifier circuit.}
\end{table}

\begin{figure}[hbtp]
  \centering
  % GNUPLOT: LaTeX picture
\setlength{\unitlength}{0.240900pt}
\ifx\plotpoint\undefined\newsavebox{\plotpoint}\fi
\sbox{\plotpoint}{\rule[-0.200pt]{0.400pt}{0.400pt}}%
\begin{picture}(1500,900)(0,0)
\sbox{\plotpoint}{\rule[-0.200pt]{0.400pt}{0.400pt}}%
\put(171.0,131.0){\rule[-0.200pt]{4.818pt}{0.400pt}}
\put(151,131){\makebox(0,0)[r]{ 0}}
\put(1419.0,131.0){\rule[-0.200pt]{4.818pt}{0.400pt}}
\put(171.0,212.0){\rule[-0.200pt]{4.818pt}{0.400pt}}
\put(151,212){\makebox(0,0)[r]{ 0.5}}
\put(1419.0,212.0){\rule[-0.200pt]{4.818pt}{0.400pt}}
\put(171.0,292.0){\rule[-0.200pt]{4.818pt}{0.400pt}}
\put(151,292){\makebox(0,0)[r]{ 1}}
\put(1419.0,292.0){\rule[-0.200pt]{4.818pt}{0.400pt}}
\put(171.0,373.0){\rule[-0.200pt]{4.818pt}{0.400pt}}
\put(151,373){\makebox(0,0)[r]{ 1.5}}
\put(1419.0,373.0){\rule[-0.200pt]{4.818pt}{0.400pt}}
\put(171.0,454.0){\rule[-0.200pt]{4.818pt}{0.400pt}}
\put(151,454){\makebox(0,0)[r]{ 2}}
\put(1419.0,454.0){\rule[-0.200pt]{4.818pt}{0.400pt}}
\put(171.0,534.0){\rule[-0.200pt]{4.818pt}{0.400pt}}
\put(151,534){\makebox(0,0)[r]{ 2.5}}
\put(1419.0,534.0){\rule[-0.200pt]{4.818pt}{0.400pt}}
\put(171.0,615.0){\rule[-0.200pt]{4.818pt}{0.400pt}}
\put(151,615){\makebox(0,0)[r]{ 3}}
\put(1419.0,615.0){\rule[-0.200pt]{4.818pt}{0.400pt}}
\put(171.0,695.0){\rule[-0.200pt]{4.818pt}{0.400pt}}
\put(151,695){\makebox(0,0)[r]{ 3.5}}
\put(1419.0,695.0){\rule[-0.200pt]{4.818pt}{0.400pt}}
\put(171.0,776.0){\rule[-0.200pt]{4.818pt}{0.400pt}}
\put(151,776){\makebox(0,0)[r]{ 4}}
\put(1419.0,776.0){\rule[-0.200pt]{4.818pt}{0.400pt}}
\put(171.0,131.0){\rule[-0.200pt]{0.400pt}{4.818pt}}
\put(171,90){\makebox(0,0){ 1}}
\put(171.0,756.0){\rule[-0.200pt]{0.400pt}{4.818pt}}
\put(330.0,131.0){\rule[-0.200pt]{0.400pt}{4.818pt}}
\put(330,90){\makebox(0,0){ 1.5}}
\put(330.0,756.0){\rule[-0.200pt]{0.400pt}{4.818pt}}
\put(488.0,131.0){\rule[-0.200pt]{0.400pt}{4.818pt}}
\put(488,90){\makebox(0,0){ 2}}
\put(488.0,756.0){\rule[-0.200pt]{0.400pt}{4.818pt}}
\put(647.0,131.0){\rule[-0.200pt]{0.400pt}{4.818pt}}
\put(647,90){\makebox(0,0){ 2.5}}
\put(647.0,756.0){\rule[-0.200pt]{0.400pt}{4.818pt}}
\put(805.0,131.0){\rule[-0.200pt]{0.400pt}{4.818pt}}
\put(805,90){\makebox(0,0){ 3}}
\put(805.0,756.0){\rule[-0.200pt]{0.400pt}{4.818pt}}
\put(964.0,131.0){\rule[-0.200pt]{0.400pt}{4.818pt}}
\put(964,90){\makebox(0,0){ 3.5}}
\put(964.0,756.0){\rule[-0.200pt]{0.400pt}{4.818pt}}
\put(1122.0,131.0){\rule[-0.200pt]{0.400pt}{4.818pt}}
\put(1122,90){\makebox(0,0){ 4}}
\put(1122.0,756.0){\rule[-0.200pt]{0.400pt}{4.818pt}}
\put(1281.0,131.0){\rule[-0.200pt]{0.400pt}{4.818pt}}
\put(1281,90){\makebox(0,0){ 4.5}}
\put(1281.0,756.0){\rule[-0.200pt]{0.400pt}{4.818pt}}
\put(1439.0,131.0){\rule[-0.200pt]{0.400pt}{4.818pt}}
\put(1439,90){\makebox(0,0){ 5}}
\put(1439.0,756.0){\rule[-0.200pt]{0.400pt}{4.818pt}}
\put(171.0,131.0){\rule[-0.200pt]{0.400pt}{155.380pt}}
\put(171.0,131.0){\rule[-0.200pt]{305.461pt}{0.400pt}}
\put(1439.0,131.0){\rule[-0.200pt]{0.400pt}{155.380pt}}
\put(171.0,776.0){\rule[-0.200pt]{305.461pt}{0.400pt}}
\put(30,453){\makebox(0,0){$V_{DC} (V)}}
\put(805,29){\makebox(0,0){$V_S (V_{peak})$}}
\put(805,838){\makebox(0,0){Peak Voltage vs. DC Voltage in Rectifier Circuit}}
\put(171,200){\usebox{\plotpoint}}
\multiput(171.00,200.58)(1.184,0.499){265}{\rule{1.046pt}{0.120pt}}
\multiput(171.00,199.17)(314.828,134.000){2}{\rule{0.523pt}{0.400pt}}
\multiput(488.00,334.58)(1.117,0.499){281}{\rule{0.993pt}{0.120pt}}
\multiput(488.00,333.17)(314.939,142.000){2}{\rule{0.496pt}{0.400pt}}
\multiput(805.00,476.58)(1.381,0.499){227}{\rule{1.203pt}{0.120pt}}
\multiput(805.00,475.17)(314.504,115.000){2}{\rule{0.601pt}{0.400pt}}
\multiput(1122.00,591.58)(1.133,0.499){277}{\rule{1.006pt}{0.120pt}}
\multiput(1122.00,590.17)(314.913,140.000){2}{\rule{0.503pt}{0.400pt}}
\put(171,200){\makebox(0,0){$+$}}
\put(488,334){\makebox(0,0){$+$}}
\put(805,476){\makebox(0,0){$+$}}
\put(1122,591){\makebox(0,0){$+$}}
\put(1439,731){\makebox(0,0){$+$}}
\put(171.0,131.0){\rule[-0.200pt]{0.400pt}{155.380pt}}
\put(171.0,131.0){\rule[-0.200pt]{305.461pt}{0.400pt}}
\put(1439.0,131.0){\rule[-0.200pt]{0.400pt}{155.380pt}}
\put(171.0,776.0){\rule[-0.200pt]{305.461pt}{0.400pt}}
\end{picture}

  \caption{\label{fig:rect_vp_vdc} AC input vs. DC output of rectifier circuit, where $R_L=\SI{10}{\kilo\ohm}$}
\end{figure}

\begin{table}
  \centering
  \begin{tabular}{ccc}
    $R_L$ & $V_{OC}$ & $V_S$ Drop \\
    (\si{\ohm})  & (\si{V}) & (\si{V}) \\
    \hline
    100 & 6.12 & 7.5 \\
    330 & 7.88 & 5.8 \\
    1k & 8.90 & 5.3 \\
  \end{tabular}
  \caption{\label{tab:volt_reg_calc} Calculated values for voltage regulator circuit}
\end{table}

\begin{table}
  \centering
  \begin{tabular}{cccccc}
    $R_L$ & $V_L$ & $I_L$ & $V_{OC}$ & $V_S$ Drop & Voltage \\
    (\si{\ohm}) & (\si{V}) & (\si{mA}) & (\si{V}) & (\si{V}) & Regulation \\
    \hline
    100 & 5.11 & 50.9 & 6.10 & 7.5 & 4.20\%  \\
    330 & 5.318 & 15.62 & 7.87 & 5.9 & 1.17\% \\
    1k & 5.163 & 5.27 & 8.60 & 5.3 & 5.28\% \\
    $\infty$ & 5.38 & --- & --- & --- & ---\\
  \end{tabular}
  \caption{\label{tab:volt_reg_meas} Measured values for voltage regulator circuit}
\end{table}

\begin{table}
  \centering
  \begin{tabular}{ccccc}
    $R_L$ & $V_{OC}$ & $V_S$ Drop \\
    (\si{\ohm}) & (\% diff) & (\% diff) \\
    \hline
    100 & 0.359\% & 0.0\% \\
    330 & 0.102\% & 1.7\% \\
    1k & 3.327\% & 0.0\% \\
  \end{tabular}
  \caption{\label{tab:volt_reg_diff} Comparison of values for voltage regulator circuit}
\end{table}

\section{Conclusion}
\label{sec:conclusion}

% As seen in Figure~\ref{fig:part_a_graph2}, the graph of the natural
% log of $I_d$ vs. $V_d$ derived from Part A data
% (Table~\ref{tab:part_a}) generates a linear plot.  The slope ($m$) of
% this line was then calculated and used to determine the thermal
% voltage ($V_T$) (Eq~\ref{eq:m}).  Two corresponding $I_d$ and $V_d$
% values (shown in Table~\ref{tab:analysis}) along with $V_T$ were
% plugged into the Schockley equation (Eq~\ref{eq:schockley}) to derive
% the saturation current ($I_S$), seen in (Table~\ref{tab:analysis}).
% The value of $V_T$ is very close to the assumed value of
% \SI{0.026}{V}.  Also the value of $I_S$ seems to be close to what is
% typically seen in circuits textbooks, thus showing that diode
% parameters can be calculated with the Schockley equation.

\section{Equations}
\label{sec:equations}

% LaTeX sees blank lines as a start of another paragraph.  To avoid
% unnecessary vertical spaces between equations, and still visually
% separate in source, put a comment between them.
%
\begin{equation}
  \label{eq:percent_diff}
  \%_{diff} = \frac{|nominal - measured|}{nominal}\times 100\%
\end{equation}
%
\begin{equation}
  \label{eq:volt_reg}
  \%_{reg} = \frac{V_{load} - V_{no load}}{V_{no load}}\times 100\%
\end{equation}

\end{document}
